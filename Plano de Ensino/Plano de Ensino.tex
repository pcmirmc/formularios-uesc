\documentclass[12pt,a4paper]{article}
\usepackage{amsmath,amsfonts,amscd,bezier,amssymb,babel}
\usepackage[utf8]{inputenc}
\usepackage{graphicx,xr,float,geometry,url}
\usepackage{answers,mathrsfs,setspace,multirow,multicol}

%=====================================================================================
\setlength{\textwidth}{14cm}
\renewcommand{\baselinestretch}{1.5}%dá o espaçamento entre linhas
\parindent 1.5cm%dá o tamanho do parágrafo
\geometry{a4paper,left=2cm,right=2.5cm,top=2.5cm,bottom=2.5cm}

%==========================================================================================

\begin{document}

\begin{center}
  \begin{tabular}{|c|c|}
    \hline
    \multirow{4}{*}{ \includegraphics[ width=1.5cm]{uesct}}   & \textbf{UNIVERSIDADE ESTADUAL DE SANTA CRUZ - UESC}\\
                                                              & \textbf{PRÓ-REITORIA DE GRADUAÇÃO - PROGRAD}\\
                                                              & \small{\textbf{DEPARTAMENTO DE CIÊNCIAS EXATAS - DCEX}}\\
                                                              & \small{\textbf{CURSO: Matemática}} \\
    \hline
  \end{tabular}
\end{center}

\vspace{1cm}

\begin{center}
  \large{\textbf{PLANO DE ENSINO}}
\end{center}

\vspace{1cm}

\hspace{-1.6cm}
\begin{tabular}{|p{4cm}|p{4cm}|p{8cm}|}
  \hline
  \hfil\textbf{CÓDIGO} &\hfil\textbf{SEMESTRE} &\hfil \textbf{DISCIPLINA} \\ \hline
  \hfil CET 167        &\hfil 2023.1           &\hfil Cálculo Diferencial e Integral III              \\ \hline

\end{tabular}\\ \vspace{-0.7cm}
\hspace{-0.17cm}
\begin{tabular}{|p{3cm}|p{3.55cm}|p{4cm}|p{5cm}|}
  \hline
  \hfil C/HORÁRIA   & \hfil CRÉDITOS &\hfil PERÍODO                     &\hfil  PROFESSOR                 \\ \hline
  \hfil T- 90       & \hfil 6        &\hfil \multirow{2}{*}{  2023.1}   &\hfil Fulano de Tal              \\ \cline{1-2}
  \hfil P- 0        &\hfil 0         &                                  &                                 \\ \hline
  \hfil  Total- 90                   &\hfil  6                          &  \multicolumn{2}{|l|}{Ass.}     \\ \hline
\end{tabular}

\vspace{1.7cm}\hspace{-1.65cm}
\begin{tabular}{|p{16.8cm}|}
  \hline
  {\bf Ementa:}

  Integrais múltiplas e aplicações. Funções vetoriais. Campos vetoriais: integrais de linha e de superfícies. Teorema de Green. Teorema de Gauss. Teorema de Stokes\\ \hline
  %=====================================
  {\bf Metodologia:}

  A disciplina de Cálculo Diferencial e Integral III é fundamental para os cursos da área de Ciências Exatas. Seu objetivo é familiarizar o aluno com as operações de Integrais múltiplas, integrais de linha e suas aplicações em problemas práticos. O conteúdo será apresentado em aulas expositivas e participativas, utilizando recursos como quadro branco, softwares de geometria dinâmica e solvers. Os alunos serão incentivados a resolver exercícios para consolidar o aprendizado.\\ \hline
  %=====================================
  {\bf Avaliações}:
  {

  - O curso será avaliado através de 4 provas, cada uma com valor de 10 pontos, 4 testes valendo 2,5 pontos cada e um seminário valendo 10 pontos.

  - As solicitações para realização de avaliações de segunda chamada devem ser feitas seguindo o protocolo estabelecido no regimento da Uesc.

  - As provas finais serão realizadas no dia XY/wz/2023.
  }
  \\ \hline
\end{tabular}

\newpage

\hspace{-1.65cm}
\begin{tabular}{|p{16.8cm}|}
\hline
%=====================================
\begin{center}
  {\bf Conteúdo Programático}
\end{center}\\
\hline
%=====================================
1ª Nota: \textbf{Integrais Múltiplas-Integrais Duplas}
\begin{itemize}

  \item Volume de sólidos obtidos por rotação em torno de um eixo
  \item Volume de um Sólido Qualquer
  \item Integral Dupla
  \begin{itemize}
    \item Integral Dupla sobre Retângulos.
    \item Integral Dupla sobre regiões não retangulares.
    \item Inversão na ordem de integração.
  \end{itemize}

\end{itemize}

2ª Nota: \textbf{Integrais Múltiplas-Integrais Triplas}

\begin{itemize}
  \item Mudança de variável na Integral Dupla, Coordenadas Polares.
  \item Conjunto de Conteúdo nulo.
  \item Integrais Triplas; Mudança de variável na Integral tripla;
  \item Coordenadas Cilíndricas e Esféricas.
  \item Centro de Massa
  \item Momento de Inércia.
\end{itemize}

3ª Nota:\textbf{Funções Vetoriais}

\begin{itemize}
  \item Funções de Várias Variáveis Reais a valores Vetoriais.
  \item Campos Vetoriais.
  \item Torção, Curvatura
  \item Comprimento de arco e parametrização de curvas.
  \item Rotacional.
\end{itemize}
\\ \hline
\end{tabular}

\newpage
\vspace{0.7cm}\hspace{-1.7cm}\begin{tabular}{|p{16.8cm}|}
\hline
\begin{itemize}
  \item Divergente.
  \item Limite e continuidade.
  \item Campo Conservativo.
  \item Forma Diferencial Exata.
  \item A integral de Linha.
  \item A integral de Linha de um campo conservativo.
  \item Condições para um campo vetorial  ser conservativo.
\end{itemize}

4ª Nota:\textbf{Integrais de linha; Campos Conservativos}

\begin{itemize}
  \item Teorema de Green.
  \item Teorema de Stokes no Plano.
  \item  Superfícies Regulares.
  \item Área de uma superfície.
  \item Integral de Superfície.
  \item Fluxo de um campo vetorial
  \item Teorema da divergência ou de Gauss
  \item Circulação e o Teorema de Stokes no espaço
\end{itemize}

5ª Nota: Soma dos quatro testes realizados

6ª Nota: Seminário sobre aplicações do Cálculo Diferencial  integral III.

\\
\hline
\end{tabular}
\newpage
\vspace{0.7cm}\hspace{-1.7cm}
\begin{tabular}{|p{16.8cm}|}
\hline
%=====================================
\begin{center}
  {\bf Referências Básicas}
\end{center}\\
%=====================================
\begin{itemize}
  \item THOMAS, George B. Cálculo. Vol 2, 11ª Edição. São Paulo: Pearson, 2008
  Disponível na Biblioteca virtual da UESC

  \url{https://plataforma.bvirtual.com.br/Leitor/Publicacao/27/pdf/0}


  \item GONÇALVES, Miriam Buss. FLEMMING, Diva M. Cálculo B. São Paulo: Pearson, 2007
  Disponível na Biblioteca virtual da UESC

  \url{https://plataforma.bvirtual.com.br/Leitor/Publicacao/413/pdf/0}


  \item RODRIGUES, Guilherme Lemermeier. Cálculo Diferencial e Integral II. Curitiba: Intersaberes: 2017
  Disponível na Biblioteca virtual da UESC

  \url{https://plataforma.bvirtual.com.br/Leitor/Publicacao/129465/pdf/0}

\end{itemize}

\\
\hline
\end{tabular}

\end{document}
