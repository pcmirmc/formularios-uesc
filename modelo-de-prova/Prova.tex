
%%%%%%%%%%%%%%%%%%%%%%%%%%%%%%%%%%%%%%%%%%%%%%%%%%%%%%%%%%%%%%%%%%%%%%%%%%%%%%%%%%%%%
%
% See http://www-math.mit.edu/~psh/#ExamCls for full documentation, but the questions
% below give an idea of how to write questions [with parts] and have the points
% tracked automatically on the cover page.
%
%
%%%%%%%%%%%%%%%%%%%%%%%%%%%%%%%%%%%%%%%%%%%%%%%%%%%%%%%%%%%%%%%%%%%%%%%%%%%%%%%%%%%%%

\documentclass[11pt]{exam}
\usepackage{amssymb, amsfonts, amsmath, latexsym, verbatim, xspace, setspace}
\usepackage[margin=1in]{geometry}
\usepackage[brazil]{babel}
\usepackage[utf8]{inputenc}
\usepackage{graphicx}
\usepackage{hyperref}
\usepackage{xcolor}

\pointname{Pontos}

\DeclareMathOperator{\sen}{sen}
\DeclareMathOperator{\arcsen}{arcsen}

\newcommand{\class}{Cálc. Dif. e Integral III}
\newcommand{\term}{2023.2}
\newcommand{\examnum}{Prova I}
\newcommand{\examdate}{13/09/2023 - 2023.2}
\newcommand{\timelimit}{100 Minutos}

\singlespacing
\parindent 0ex
%============================================================================================================
\begin{document} 

\pagestyle{head}
\firstpageheader{}{}{}
\runningheader{\class}{\examnum\ - Página \thepage\ de \numpages}{\examdate}
\runningheadrule

\begin{center}
  \begin{minipage}{2.5cm}
    \includegraphics[width=1.5cm]{uesct}
  \end{minipage}
\end{center}

\begin{minipage}[b]{1.65\linewidth}
  \begin{tabular}{c}
    \hline \hline
    {\bf UNIVERSIDADE ESTADUAL DE SANTA CRUZ} \\
    {\bf DEPARTAMENTO DE CIÊNCIAS EXATAS - DCEX} \\
    {\bf Cálculo Diferencial e Integral III} \\
    {\bf Prova } \\
    \\
    \hline
    {\small \textbf{Início: 07:30}} \hfill {\small \textbf{Término: 09:10} } \\
    \hline
    {\small \textbf{Professor:} Fulano de Tal} \hfill 13 de Setembro de 2023 \\
    \hline
    {\small \textbf{Aluno(a):}} \hspace{10.0cm} \textbf{Matrícula:} \hspace{2.3cm} \\
    \hline
  \end{tabular}
\end{minipage} \hfill

\vspace{1cm}

Este exame contém \numpages\ páginas (incluindo a capa) e \numquestions\ questão(ões). Verifique se falta alguma página. 
Você precisa mostrar seus conhecimentos em cada problema deste exame. As seguintes regras se aplicam:\\

\begin{minipage}[t]{3.7in}
  \vspace{0pt}
  \begin{itemize}
    \item Pontuação máxima a ser obtida é de 10(dez) pontos.
    \item \textbf{Se você usar um ``Teorema Fundamental" você deve indicar isto} e explicar por quê o teorema pode ser aplicado;

    \item \textbf{Organize seu trabalho} de forma razoavelmente limpa e coerente. Questões sem ordem clara de ideias perderão pontuação.

    \item Lembre-se que o objetivo nesta avaliação é analisar sua forma de argumentação e escrita, o seu raciocínio. Não tente ``enrolar".

    \item \textbf{Respostas misteriosas ou sem justificativa não receberão pontuação completa}. Resposta correta, não justificada por cálculos, explicações ou trabalho algébrico não receberá pontuação; Respostas incorretas, justificadas por cálculos corretos, podem ainda receber pontuação.

    \item Prazo final é prazo final. Avaliações entregues após o horário limite não receberão pontuação.

  \end{itemize}

\end{minipage}
\hfill
\begin{minipage}[t]{2.3in}
  \vspace{0pt}
  \gradetablestretch{2}
  \vqword{Questão}
  \vpword{Pontos}
  \addpoints
  \gradetable[v]
\end{minipage}

\newpage 

Considerando os dígitos do seu número de \textbf{matrícula 20xyzw\textcolor{blue}{\textbf{abc}}, e tomando os três últimos dígitos(abc)}, responda às questões  abaixo:

\begin{questions}
  \large
  \addpoints

  \question[2\half] A região delimitada pelo gráfico da função $f$ e o eixo $x$ gira em torno da reta r . Calcule o volume do sólido obtido.

  (Dados da questão:\url{https://www.geogebra.org/classic/mdn6anqn} )
  \vfill

  \question[2\half] Utilizando integrais duplas, calcule o volume do sólido que está acima do plano xy, cuja base é a região entre a reta $y=(\frac{a}{3}-2)x$ e a parábola $y=x^2-4x$  e o topo é o plano $f(x,y)=-x-y+6$.

  \vfill

  \question[2\half] Inverta a ordem de integração da questão anterior.\textit{(\textbf{Não} é necessário recalcular a integral).}
  \vfill

  \question[2\half] O sólido situa-se entre os planos perpendiculares ao eixo x em $x=b+1$ e $x=-b-1$. As seções transversais são \textbf{triângulos equiláteros} perpendiculares ao plano xy, cujas bases são paralelas ao eixo y, e vão da borda inferior a superior da circunferência $$x^2+y^2=(c+b+1)^2$$ Determine o volume do sólido.

  \vfill

\end{questions}

\end{document}
